\documentclass[10pt]{article}
% \usepackage{fontspec}
% \usepackage{unicode-math}
\usepackage[utf8]{inputenc}
\usepackage[a4paper,hmargin={4cm,4cm},vmargin={3.5cm,3.5cm}]{geometry}
\usepackage[colorlinks=true,bookmarksopen=false,linkcolor=blue,citecolor=red]{hyperref}
\usepackage{subfiles}
\usepackage{calc} 
\usepackage{datetime}
\usepackage{comment}
\usepackage{amssymb}
\usepackage{amsthm}
\usepackage{amsmath}
\usepackage{amsrefs}
\usepackage{titlesec}
\usepackage{titletoc}
\usepackage{dsfont}
\usepackage{euscript}
\usepackage{fourier-orns}
%\usepackage{pxfonts}
%\usepackage{newpxtext}
%\usepackage{tgpagella}
\usepackage{palatino}
\usepackage[sc]{mathpazo} % add possibly `sc` and `osf` options
%\usepackage{eulervm}
%\usepackage[adobe-utopia]{mathdesign}
\usepackage[T1]{fontenc}
\usepackage{graphicx}
\usepackage{tikz}
% \usetikzlibrary{tikzmark}
% \usepackage{tikz-cd}
% \tikzcdset{
% arrow style=tikz,
% diagrams={>=latex}
% }

\linespread{1.265}
\setlength{\parindent}{0ex}
\setlength{\parskip}{.4\baselineskip}
\definecolor{blue}{rgb}{0, 0.1, 0.6}

\DeclareFontFamily{OT1}{pzc}{}
\DeclareFontShape{OT1}{pzc}{m}{it}{<-> s * [1.10] pzcmi7t}{}
\DeclareMathAlphabet{\mathpzc}{OT1}{pzc}{m}{it}

\newcommand{\mylabel}[1]{{\ssf{#1}}\hfill}
\renewenvironment{itemize}
  {\begin{list}{$\triangleright$}{%
   \setlength{\parskip}{0mm}
   \setlength{\topsep}{.4\baselineskip}
   \setlength{\rightmargin}{0mm}
   \setlength{\listparindent}{0mm}
   \setlength{\itemindent}{0mm}
   \setlength{\labelwidth}{3ex}
   \setlength{\itemsep}{.4\baselineskip}
   \setlength{\parsep}{0mm}
   \setlength{\partopsep}{0mm}
   \setlength{\labelsep}{1ex}
   \setlength{\leftmargin}{\labelwidth+\labelsep}
   \let\makelabel\mylabel}}{%
   \end{list}\vspace*{-\parskip}}

\def\E{\exists}
\def\A{\forall}
\def\Indicator{{\mathds 1}}
\def\mdot{\mathord\cdot}
\def\models{\vDash}
\def\pmodels{\mathrel{\models\kern-1.5ex\raisebox{.5ex}{*}}}
\def\notmodels{\nvDash}
\def\proves{\vdash}
\def\notproves{\nvdash}
\def\proves{\vdash}
\def\provesT{\mathrel{\mathord{\vdash}\hskip-1.13ex\raisebox{-.5ex}{\tiny$T$}}}
\def\ZZ{\mathds Z}
\def\NN{\mathds N}
\def\QQ{\mathds Q}
\def\RR{\mathds R}
\def\BB{\mathds B}
\def\CC{\mathds C}
\def\PP{\mathds P}
\def\Ar{{\rm Ar}}
\def\dom{\mathop{\rm dom}}
\def\supp{\mathop{\rm supp}}
\def\range{\mathop{\rm img}}
\def\rank{\mathop{\rm rank}}
\def\dcl{\mathop{\rm dcl}}
\def\acl{\mathop{\rm acl}}
\def\fp{{\rm fp}}
\def\cf{\mathop{\rm cf}}
\def\rad{\mathop{\rm rad}}
\def\eq{{{\rm eq}}}
\def\ccl{{\rm ccl}}
\def\Th{\textrm{Th}}
\def\Diag{{\rm Diag}}
\def\atpmTh{\textrm{Th}_\atpm}
\def\Mod{\mathop{\rm Mod}}
\def\Fr{\mathop{\rm Fr}}
\def\Rmod{{\mbox{\scriptsize $R$-mod}}}
\def\Aut{{\rm Aut\kern.15ex}}
\def\Autf{\mathord{\rm Aut\kern.15ex{f}\kern.15ex}}
\def\Cb{\textrm{Cb\kern.15ex}}
\def\orbit{\O}
\def\oorbit{\mathpzc{o}}
\def\oorbitf{\mathpzc{of\!}}
\def\id{{\rm id}}
\def\tp{{\rm tp}}
\def\atpm{{\tiny\rm at^\pm}}
\def\qftp{\textrm{qf\mbox{-}tp}}
\def\attp{\textrm{at\mbox{-}tp}}
\def\atpmtp{\atpm\mbox{-}\textrm{tp}}
\def\Deltatp{\Delta\mbox{-}\textrm{tp}}
\def\pmDelta{\Delta\hskip-.3ex\raisebox{1ex}{\tiny$\pm$}}
\def\pmDeltatp{\noindent\pmDelta\hskip-.3ex\textrm{-tp}}
\def\EMtp{\textrm{{\small EM}\mbox{-}tp}}

\newcommand{\cev}[1]{\reflectbox{\ensuremath{\vec{\reflectbox{\ensuremath{#1}}}}}}
\newcommand{\sbar}[1]{\mkern 1.8mu\overline{\mkern-1.8mu#1\mkern-1mu}\mkern 1mu}

\def\nonfork{\mathop{\raise0.2ex\hbox{
   \ooalign{\hidewidth$\vert$\hidewidth\cr\raise-0.9ex\hbox{$\smile$}}}}}

\def\cnonfork{\mathbin{\raise1.8ex\rlap{\kern0.6ex\rule{0.6ex}{0.1ex}}
\rlap{\kern1.1ex\rule{0.1ex}{1.9ex}}\raise-0.3ex\hbox{$\smile$} } }

\def\cpaw{\mathbin{\ooalign{\kern-0.4ex$-$\hidewidth\cr$<$}}}
\def\cpawdot{\ooalign{$\kern1.2ex\cdot$\cr$\cpaw$\cr}}

\def\sm{\smallsetminus}
\def\Trm{{\rm Trm}}
\def\IMP{\Rightarrow}
\def\PMI{\Leftarrow}
\def\IFF{\Leftrightarrow}
\def\NIFF{\nLeftrightarrow}
\def\imp{\rightarrow}
\def\pmi{\leftarrow}
\def\iff{\leftrightarrow}
\def\niff{\mathrel{{\leftrightarrow}\llap{\raisebox{-.1ex}{{\small$/$}}\hskip.5ex}}}
\def\nimp{\mathrel{{\rightarrow}\llap{\raisebox{-.1ex}{{\small$/$}}\hskip.5ex}}}
\def\nequiv{\mathrel{\mbox{$\equiv$\llap{{\small$/$}\hskip.3ex}}}}

\def\isomap{\rlap{\kern0.8ex\raisebox{1ex}{\scriptsize$\sim$}}\rightarrow}

\def\P{\EuScript P}
\def\D{\EuScript D}
\def\Aa{\EuScript A}
\def\Ee{\EuScript E}
\def\X{\EuScript X}
\def\Y{\EuScript Y}
\def\Z{\EuScript Z}
\def\C{\EuScript C}
\def\U{\EuScript U}
\def\W{\EuScript W}
\def\Hh{\EuScript H}
\def\I{\EuScript I}
\def\V{\EuScript V}
\def\R{\EuScript R}
\def\F{\EuScript F}
\def\G{\EuScript G}
\def\B{\EuScript B}
\def\M{\EuScript M}
\def\N{\EuScript N}
\def\Ll{\EuScript L}
\def\K{\EuScript K}
\def\O{\EuScript O}
\def\J{\EuScript J}
\def\S{\EuScript S}
\def\T{\EuScript T}
\def\<{\langle}
\def\>{\rangle}
\def\0{\varnothing}
\def\theta{\vartheta}
\def\phi{\varphi}
\def\sharpphi{{\scriptstyle\Sigma}\kern0ex\varphi}
\def\epsilon{\varepsilon}
\def\ssf#1{\textsf{\small #1}}

\titlecontents{section}
[3.8em] % ie, 1.5em (chapter) + 2.3em
{\vskip-1ex}
{\contentslabel{1.5em}}
{\hspace*{-2.3em}}
{\titlerule*[1pc]{}\contentspage}

\titleformat{\section}[block]{\Large\bfseries}{\makebox[5ex][r]{\textbf{\thesection}}}{1.5ex}{}
\titlespacing*{\chapter}{0em}{.5ex plus .2ex minus .2ex}{2.3ex plus .2ex}
\titlespacing*{\section}{-9.7ex}{3ex plus .5ex minus .5ex}{1ex plus .2ex minus .2ex}

\newtheoremstyle{mio}% name
     {2\parskip}%      Space above
     {\parskip}%      Space below
     {\sl}%         Body font
     {}%          Indent amount (empty = no indent, \parindent = para indent)
     {\bfseries}% Thm head font
     {}%        Punctuation after thm head
     {1ex}%     Space after thm head: " " = normal interword space;
         %   \newline = linebreak
     {\llap{\thmnumber{#2}\hskip2mm}% Thm head spec (empty means `normal')
      \thmname{#1}\thmnote{\bfseries{}#3}}

\newtheoremstyle{liscio}% name
     {2\parskip}%      Space above
     {0mm}%      Space below
     {}%         Body font
     {}%         Indent amount (empty = no indent, \parindent = para indent)
     {\bfseries}% Thm head font
     {}%        Punctuation after thm head
     {1.5ex}%     Space after thm head: " " = normal interword space;
            %   \newline = linebreak
     {\llap{\thmnumber{#2}\hskip2mm}% Thm head spec (empty means `normal')
      \thmname{#1}\thmnote{\bfseries{} #3}}

\newcounter{thm}[section]
\renewcommand{\thethm}{\thesection.\arabic{thm}}

\theoremstyle{mio}
\newtheorem{theorem}[thm]{Theorem}
\newtheorem{corollary}[thm]{Corollary}
\newtheorem{proposition}[thm]{Proposition}
\newtheorem{lemma}[thm]{Lemma}
\newtheorem{fact}[thm]{Fact}
\newtheorem{definition}[thm]{Definition}
\newtheorem{assumption}[thm]{Assumption}
\newtheorem{void_thm}[thm]{}
\theoremstyle{liscio}
\newtheorem{void_def}[thm]{}
\newtheorem{remark}[thm]{Remark}
\newtheorem{notation}[thm]{Notation}
\newtheorem{note}[thm]{Note}
\newtheorem{exercise}[thm]{Exercise}
\newtheorem{example}[thm]{Example}
\setlength{\partopsep}{0mm}
\setlength{\topsep}{0mm}

\def\QED{\noindent\nolinebreak[4]\hspace{\stretch{1}}\rlap{\ \ $\Box$}\medskip}
\renewenvironment{proof}[1][Proof]%
{\begin{trivlist}\item[\hskip\labelsep {\bf #1}]}
{\QED\end{trivlist}}

\newenvironment{void}[1][]%
{\begin{trivlist}\item[\hskip\labelsep {\bf #1}]}
{\QED\end{trivlist}}

\pagestyle{plain}

\definecolor{violet}{RGB}{115, 0, 205}
\definecolor{brown}{RGB}{150, 50, 10}
\definecolor{green}{RGB}{5,110, 35}
\definecolor{emphcolor}{rgb}{.90,.98,.90}

\def\bl{\color{black}}
%\def\bl{\color{brown}}
\def\mr{\color{brown}}
\def\gr{\color{green}}
\def\vl{\color{violet}}

\def\mrA{{\mr\Aa}}
\def\mrB{{\mr\B}}
\def\mrC{{\mr\C}}
\def\mrD{{\mr\D}}
\def\mrE{{\mr\Ee}}
\def\mrG{{\mr\G}}
\def\mrU{{\mr\U}}
\def\mrV{{\mr\V}}
\def\mrS{{\mr\S}}
\def\mrP{{\mr\P}}
\def\mrW{{\mr\W}}
\def\grB{{\gr\B}}
\def\grC{{\gr\C}}
\def\grD{{\gr\D}}
\def\grV{{\gr\V}}

\renewcommand*{\emph}[1]{%
   \kern-0.2ex 
   \smash{\tikz[baseline]
   \node[ rectangle, fill=emphcolor, rounded corners, 
          inner xsep=.3ex, inner ysep=.2ex, anchor=base,
          minimum height = 3ex
         ]{#1};
   }
   \kern-1.2ex 
}

\begin{document}
\raggedbottom
\begin{center}
   {\huge\bfseries Scratch paper\\[3ex] \normalfont\normalsize 
   Anonymous\vskip-1ex 
   Università di Torino\vskip-1ex \monthname\ \the\year}
\end{center}

\def\medrel#1{\parbox[t]{6ex}{$\displaystyle\hfil #1$}}

\bigskip\hfil
\parbox{0.9\textwidth}{
   \textbf{Abstract} \ 
   Poche idee, ben confuse.
}

\bigskip{\tt branch: Szemeredi}
%%%%%%%%%%%%%%%%%%%%%%%%%%%%%%%%%%%%%%%%%%%%%%%%%%
%%%%%%%%%%%%%%%%%%%%%%%%%%%%%%%%%%%%%%%%%%%%%%%%%%
%%%%%%%%%%%%%%%%%%%%%%%%%%%%%%%%%%%%%%%%%%%%%%%%%%
%%%%%%%%%%%%%%%%%%%%%%%%%%%%%%%%%%%%%%%%%%%%%%%%%%
%%%%%%%%%%%%%%%%%%%%%%%%%%%%%%%%%%%%%%%%%%%%%%%%%%
%%%%%%%%%%%%%%%%%%%%%%%%%%%%%%%%%%%%%%%%%%%%%%%%%%
%%%%%%%%%%%%%%%%%%%%%%%%%%%%%%%%%%%%%%%%%%%%%%%%%%
%%%%%%%%%%%%%%%%%%%%%%%%%%%%%%%%%%%%%%%%%%%%%%%%%%
\section{Abstract samples}\label{samples}
\def\medrel#1{\parbox[t]{6ex}{\hfil$#1$}}
\def\ceq#1#2#3{\parbox[t]{19ex}{$\displaystyle #1$}\medrel{#2}$\displaystyle  #3$}
\def\uuu{{\mathds 1}}
\def\gruuu{{\mathds 1}}
\def\Av{{\rm Av}}
\def\st{{\rm st}}

Let $M$ be a structure.
%
The \emph{finite (fractional) sample expansion\/} of $M$ is a 3-sorted expansion $\bar M=\<M,\RR,M^{\sf s}\>$ where
\begin{itemize}
  \item[1.] $M$ is called the \emph{home-sort;}
  \item[2.] $\RR$ is the set of real numbers in the signature of ordered field;
  \item[3.] $M^{\sf s}$ is an vector lattice space (a Riesz space) containing all functions $s:M\to\RR$ with finite support (i.e.\@ almost always zero); this we call the \emph{sample-sort.}
\end{itemize}

The elements $M^{\sf s}$ maybe thought as (signed) discrete measures concentrated on the support.

Variables of sample-sort (or tuple thereof) are denoted with symbols $x^{\sf s}, y^{\sf s},\dots,$ and variations.
Elements of sample-sort are denoted with the symbols $s,r,t$, and variations.
%
The variable $x$ is reserved for (tuples of) variables of the home-sort.
The context will clarify the sort the other variables.

The language of $\bar M$ is denoted by \emph{$L$.}
It expands the language of the three sorts above with a function symbol for each formula $\phi(x\,; y)\in L$, where $y$ is a tuple a mixed sort. This is interpreted with function $f_{\phi(x\,; y)}:(M^s)^{|x|}\times M^{y}\to\RR$ as explained below.

For legibility, substitute parameters for $y$. 
So, for $\phi(x)\in L(\bar M)$ we define\medskip

\ceq{\hfill f_{\phi(x)}(s)}
{=}
{\sum_{a\,\models\,\phi(x)}\kern-.8ex s(a)}
\smallskip

\ceq{\hfill \textrm{(or, for short)}}
{=}
{\ \ \emph{$\displaystyle\sum_{\phi(x)} s$}}
\bigskip

As usual with vector lattices, we write $s^+=s\vee 0$ for the nonnegative part of $s$ and $s^-=s\wedge 0$ for the nonpositive part of $s$. 
Finally we define $|s|=s^++s^-$.
The norm of $s$ is 


\ceq{\hfill\emph{$\|s\|$}}
{=}
{\ \sum_{x=x} |s|.}

\ceq{}
{=}
{\sum_{a\in\U^{|x|}} |s(a)|.}

Finally we write 

\ceq{\hfill\emph{$\displaystyle\Fr_{s}\phi(x)$}}
{=}
{\frac{1}{\|s\|}\sum_{\phi(x)} |s|.}

When ambiguity is of concern we write \emph{$\displaystyle\Fr_{x\in s}\phi(x)$.}

% Below, $\epsilon$ always ranges over the positive standard reals.
% %
% If $a$ and $b$ are (hyper)reals, we write \emph{$a\approx_\epsilon b$\/} for $|a-b|<\epsilon$.
% %
% We write \emph{$a\approx b$\/} if $a\approx_\epsilon b$ holds for every $\epsilon$.


\begin{definition}
  Let $\U$ be a monster model of cardinality $\kappa>|L|$.
  %
  Let \emph{$\bar\U=\<\U,\RR,\U^{\sf s}\>$\/} be as above and let \emph{$\bar\U^*$} be a saturated elementary extension of $\bar\U$ of cardinality $\kappa$.
  As all saturated models of cardinality $\kappa$ are isomorphic, we can assume that $\U$ is the domain of the home-sort of $\bar\U^*$, hence we write $\bar\U^*=\<\U,\RR^*,\U^{\sf s*}\>$.  
  %
  Elements of $\U^{\sf s}$ are called \emph{finite samples,} elements of $\U^{\sf s*}$ are called \emph{hyperfinite samples.}\QED

  %Finally, a \emph{external sample\/} is a maximally consistent set of formulas of the form $\sharpphi(x^{\sf s})\approx_\epsilon\mu$, where $\phi(x)\in L(\U^{\sf s*})$, and $\epsilon,\mu\in\RR$.
  
  % Finally, an \emph{external sample\/} is a maximally consistent set of formulas $p(x^{\sf s})\subseteq L(\U,\RR,\U^{\sf s*})$. 
  % (Note these are not types over the full of $\bar\U$.)
\end{definition}

% If there is an $n\in\RR^*$ say that $\bar\U$ is a pseudofinite structure 

\begin{remark}[(pseudofinite structures)]
  Let $\{M_i:i<\omega\}$ be a set of structures and define 
  
  \ceq{\hfill M}
  {=}
  {\bigcup_{i<\omega}\{i\}\times M_i}

  We stipulate that $M\models r(a)$ if $a=\<i,a_1\>,\dots,\<i,a_n\>$ for some $i$ and $M_i\models r(a_1,\dots,a_n)$.
  Functions symbols are interpreted in a similar way (on mixed tuples function take some fixed but arbitrary value, say the first component of the tuple).
  In the signature of $M$ we include also a symbol for an equivalence relation that has $\{i\}\times M_i$ as equivalence classes.
  In $\U$, every equivalence class is a saturated elementary extension of some ultraproducts of $\{M_i:i<\omega\}$.
  When the $M_i$ have finite unbounded cardinality we call each equivalence class in $\U$ a pseudofinite structure.
  In this case each class is the support of some $0$-$1$ valued internal sample $s\in\U^{\sf s*}$.
  In particular, every class not in $M$ has (properly) hyperfinite cardinality.\QED 
\end{remark}



Hyperfinite samples that are 0-1 valued are called \emph{hyperfinite sets}.
We write $a\in s$ for $s(a)=1$, when $s$ is an hyperfinite set.

Let $s$ be a hyperfinite set and fix some formula $\phi(x\,;y)\in L(\bar\U^*)$ where $x,y$ are variables of the home-sort.
Define

\ceq{\hfill a\sim_s a'}
{=}
{\A y\in s\big[\phi(a,y)\iff\phi(a',y)\big]}




\begin{notation}
  Let $y$ and $z$ be tuples of mixed sort.
  For any type $p(y)$ and formula $\phi(y,z)\in L(\bar\U^*)$ we write 
  
  \ceq{\hfill\emph{$\phi(p\,;\bar\U^*)$}}
  {=}
  {\Big\{ a\in(\bar\U^*)^z\ :\ p(y)\proves\phi(y\,;a)\Big\}}
  
  This notation intentionally confuses $p(y)$ with any of its realizations in some elementary extension of $\bar\U$.
  \QED
\end{notation}

\begin{definition}\label{def_} 
  Let $M$ be given.
  Let $y$ be a tuple of the home-sort.
  A external sample $p(x^{\sf s})$ is
  \begin{itemize}
    \item[1.]\noindent\emph{invariant\/} if $\phi(p\,;\U,\U^{\sf s*})$ is invariant over $\bar M$, for every $\phi(x\,;y,z^{\sf s})\in L$.
    \item[2.]\noindent\emph{finitely satisfiable\/} if every formula in $p(x^{\sf s})$ is satisfied by some element of $M^{\sf s}$.
    \item[3.]\noindent\emph{definable\/} if $\phi(p\,;\U,\U^{\sf s*})$ is definable over $\bar M$, for every $\phi(x\,;y,z^{\sf s})\in L$.\QED
  \end{itemize}
  An external type is \emph{generically stable\/} if it is definable and finitely satisfiable in some $\bar M$.\QED
\end{definition}

Clearly every finitely satisfiable external sample is invariant.

We will use the symbol \emph{$\cnonfork_{\bar M}$} with the usual meaning.


\section{Type-definable functions}\label{tp-def-fun}

\def\medrel#1{\parbox[t]{6ex}{\hfil$#1$}}
\def\ceq#1#2#3{\parbox[t]{22ex}{$\displaystyle #1$}\medrel{#2}$\displaystyle  #3$}

Let $\U$ be a monster model.
Let $X$ be a Hausdorff space.
We say that the partial map $f:\U^{|x|}\to X$ is \emph{bounded\/} if $\range f\subseteq C$ for some compact $C \subseteq X$.
We say that $f$ is \emph{type-definable\/} if it is bounded and $f^{-1}[C]$ is type-definable for every compact $C\subseteq X$.
If $f^{-1}[C]$ is definable for every compact $C\subseteq X$, then we say that $f$ is \emph{definable.}

We always assume $\U$ has a larger cardinality than the topology of $X$ and that a type-definable function is always type-definable over some small set of parameters, i.e.\@ all the types defining $f^{-1}[C]$ are over the same small set parameters $A\subseteq\U$.

\begin{fact}
  Let $f:\U^{|x|}\to X$ be type-definable.
  Then the following are equivalent
  \begin{itemize}
    \item[1.] $f$ is definable;
    \item[2.] $\range f$ is finite.
  \end{itemize}
\end{fact}

\begin{proof}
  By compactness.
\end{proof}

\begin{fact}\label{fact_cover_type}
  Let $p_i(x)\subseteq L(A)$, for $i=1,\dots,n$, be a cover of a type-definable set $\D\subseteq\U^{|x|}$.
  Then there is a cover of $\D$ by some formulas $p_i(x)\proves\phi_i(x)$.
\end{fact}

\begin{proof}
  By compactness.
\end{proof}

\begin{fact}\label{fact_apx}
  Let $f:\U^{|x|}\to[0,1]$ be type-definable.
  Then for every $\epsilon>0$ there is a definable function  $g:\U^{|x|}\to[0,1]$ such that

  \ceq{\hfill \big|fa-ga\big|}{\le}{\epsilon}\hfill for every $a\in\dom f$
\end{fact}

\begin{proof}
  Pick $r_1,\dots,r_n\in[0,1]$ be such that the intervals $[r_i-\epsilon, r_i+\epsilon]$ cover $\range f$.
  Hence, the types defining $f^{-1}[r_i-\epsilon, r_i+\epsilon]$ cover $\dom f$.
  Therefore, by the fact above, there is a definable partition of $\dom f$ that refines  $f^{-1}[r_i-\epsilon, r_i+\epsilon]$.
  Define $g$ to be such that $\big\{g^{-1}[r_i]: i\in[n]\big\}$ is the partition above. 
  Then $g$ is as required.
\end{proof}

\begin{fact}
  Let $f:\U^{|x|}\to[0,1]$ be type-definable.
  Then for every $\epsilon>0$ there are finitely many $a_1,\dots,a_n\in\dom f$ such that

  \hfill$\displaystyle \bigvee_{i=1}^n\ \big|fa-f a_i\big|\le\epsilon$\hfill for every $a\in\dom f$.

  Moreover, there are some formulas $\psi_i(x)\in L(\U)$ that cover $\dom f$ and
  
  \ceq{\hfill\psi_i(x)}{\imp}{\big|fx-f a_i\big|\le\epsilon}\hfill for every $i\in[n]$.
\end{fact}

\begin{proof}
  By Fact~\ref{fact_apx} there is a definable function $g$ such that $\big|fa-ga\big|\le\epsilon/2$ for  every $a\in\dom f$.
  Pick some $a_1,\dots,a_n\in\U^{|x|}$ such that $\{ga_1,\dots,ga_n\}=\range g$.
  Then

  \ceq{\hfill \min_{i\in[n]}\big|fa-f a_i\big| }
  {=}
  {\min_{i\in[n]}\big|fa- ga + ga-fa_i\big|}
  
  \ceq{\hfill}
  {\le}
  { \big|fa-g a\big|+\min_{i\in[n]}\big|fa_i-g a\big|}

  \ceq{\hfill}
  {\le}
  { \big|fa-g a\big|+\big|fa_i-g a_i\big|}\hfill where $i$ is such that $ga_i=ga$

  \ceq{\hfill}{\le}{\epsilon}

  The second claim follows from Fact~\ref{fact_cover_type} because $|fx-fa_i|\le\epsilon$ is equivalent to a type.
\end{proof}


%%%%%%%%%%%%%%%%%%%%%%%%%%%%%%%%%%%%%%%%%%%%%%%%%%%%%%%%%%%%%%%%%%%%%%%%%%%%
%%%%%%%%%%%%%%%%%%%%%%%%%%%%%%%%%%%%%%%%%%%%%%%%%%%%%%%%%%%%%%%%%%%%%%%%%%%%
%%%%%%%%%%%%%%%%%%%%%%%%%%%%%%%%%%%%%%%%%%%%%%%%%%%%%%%%%%%%%%%%%%%%%%%%%%%%
%%%%%%%%%%%%%%%%%%%%%%%%%%%%%%%%%%%%%%%%%%%%%%%%%%%%%%%%%%%%%%%%%%%%%%%%%%%%
%%%%%%%%%%%%%%%%%%%%%%%%%%%%%%%%%%%%%%%%%%%%%%%%%%%%%%%%%%%%%%%%%%%%%%%%%%%%
%%%%%%%%%%%%%%%%%%%%%%%%%%%%%%%%%%%%%%%%%%%%%%%%%%%%%%%%%%%%%%%%%%%%%%%%%%%%
%%%%%%%%%%%%%%%%%%%%%%%%%%%%%%%%%%%%%%%%%%%%%%%%%%%%%%%%%%%%%%%%%%%%%%%%%%%%
%%%%%%%%%%%%%%%%%%%%%%%%%%%%%%%%%%%%%%%%%%%%%%%%%%%%%%%%%%%%%%%%%%%%%%%%%%%%
\section{Externally definable sets (notation)}\label{x-set}


%%%%%%%%%%%%%%%%%%%%%%%%%%%%%%%%%%%%%%%%%%%%%%%%%%%%%%%%%%%%%%%%%%%%%%%%%%%%
%%%%%%%%%%%%%%%%%%%%%%%%%%%%%%%%%%%%%%%%%%%%%%%%%%%%%%%%%%%%%%%%%%%%%%%%%%%%
%%%%%%%%%%%%%%%%%%%%%%%%%%%%%%%%%%%%%%%%%%%%%%%%%%%%%%%%%%%%%%%%%%%%%%%%%%%%
%%%%%%%%%%%%%%%%%%%%%%%%%%%%%%%%%%%%%%%%%%%%%%%%%%%%%%%%%%%%%%%%%%%%%%%%%%%%
%%%%%%%%%%%%%%%%%%%%%%%%%%%%%%%%%%%%%%%%%%%%%%%%%%%%%%%%%%%%%%%%%%%%%%%%%%%%
%%%%%%%%%%%%%%%%%%%%%%%%%%%%%%%%%%%%%%%%%%%%%%%%%%%%%%%%%%%%%%%%%%%%%%%%%%%%
%%%%%%%%%%%%%%%%%%%%%%%%%%%%%%%%%%%%%%%%%%%%%%%%%%%%%%%%%%%%%%%%%%%%%%%%%%%%
%%%%%%%%%%%%%%%%%%%%%%%%%%%%%%%%%%%%%%%%%%%%%%%%%%%%%%%%%%%%%%%%%%%%%%%%%%%%
%%%%%%%%%%%%%%%%%%%%%%%%%%%%%%%%%%%%%%%%%%%%%%%%%%%%%%%%%%%%%%%%%%%%%%%%%%%%
%%%%%%%%%%%%%%%%%%%%%%%%%%%%%%%%%%%%%%%%%%%%%%%%%%%%%%%%%%%%%%%%%%%%%%%%%%%%
%%%%%%%%%%%%%%%%%%%%%%%%%%%%%%%%%%%%%%%%%%%%%%%%%%%%%%%%%%%%%%%%%%%%%%%%%%%%
%%%%%%%%%%%%%%%%%%%%%%%%%%%%%%%%%%%%%%%%%%%%%%%%%%%%%%%%%%%%%%%%%%%%%%%%%%%%
%%%%%%%%%%%%%%%%%%%%%%%%%%%%%%%%%%%%%%%%%%%%%%%%%%%%%%%%%%%%%%%%%%%%%%%%%%%%
%%%%%%%%%%%%%%%%%%%%%%%%%%%%%%%%%%%%%%%%%%%%%%%%%%%%%%%%%%%%%%%%%%%%%%%%%%%%
%%%%%%%%%%%%%%%%%%%%%%%%%%%%%%%%%%%%%%%%%%%%%%%%%%%%%%%%%%%%%%%%%%%%%%%%%%%%
%%%%%%%%%%%%%%%%%%%%%%%%%%%%%%%%%%%%%%%%%%%%%%%%%%%%%%%%%%%%%%%%%%%%%%%%%%%%
\section{Szemer\'edi's regularity lemma}
\def\medrel#1{\parbox[t]{6ex}{\hfil$#1$}}
\def\ceq#1#2#3{\parbox[t]{29ex}{$\displaystyle #1$}\medrel{#2}$\displaystyle  #3$}

An internal nonnegative sample $s\in\U^{\sf s*}$ is pseudorandom over $M$ if for every $0\le s'\le s$ such that $\|s'\|\approx\|s\|$ and for every $\phi(x)\in L(\bar M)$

\ceq{\hfill\Fr_{s'}\phi(x)}
{\approx}
{\Fr_{s}\phi(x)}

For every $\epsilon>0$ there is an $n=n(\epsilon)$ such that for every formula $\phi(x,z)$ and every finite nonnegative sample $a,b\in M^{\sf s}$ there are $0\le s_i\le a$ and $0\le t_i\le b$, for $i\in[n]$ and a set $\Sigma\subseteq[n]^2$ such that 

and for every 

\ceq{\hfill\big|\phi(A,B)-\Fr_{s_i,t_j}\phi(x,y)\big|}
{ }
{ }


\ceq{\hfill \big|\phi(a^{\rm s}\,;b^{\rm s})\big|)}
{=}
{\sum_{\phi(x\,;y)} a^{\rm s}(x)\cdot b^{\rm s}(y)}


\ceq{\hfill d_\phi(r\,;s)}
{=}
{\frac{\big|\phi(r\,;s)\big|}{\big|r\times s\big|}}

We say that the pair of samples $r,s$ is $\epsilon$-regular if for every $r'\subseteq r$ and $s'\subseteq s$ such that  $|r'|>\epsilon|r|$ and $|s'|>\epsilon|s|$ 

\ceq{\hfill \big|d_\phi(r\,;s)-d_\phi(r';s') \big|}
{<}
{\epsilon}


\begin{void_thm}[Szemer\'edi's Regularity Lemma]
  For every $\epsilon>0$ and for every pair of samples $u,v\in\bar\U^*$ of hyperfinite cardinality there are some finite partitions of $u$ and $v$, say $r_1,\dots,r_n$ and  $s_1,\dots,s_m$, such that 

  \ssf{1.}\quad $r_i,s_j$ is $\epsilon$-regular for every $i,j\in[n]\times[m]\sm X$,

  where $X\subseteq[n]\times[m]$, called the exceptional set, is such that

  \ceq{\ssf{2.}\hfill\sum_{\<i,j\>\in X}|r_i|{\cdot}|s_j|}
  {<}
  {\epsilon{\cdot}|u|{\cdot}|v|}
\end{void_thm}

\begin{proof}
  Let $\st((d(u\,;v))$
\end{proof}


%%%%%%%%%%%%%%%%%%%%%%%%%%%%%%%%%%%%%%%%%%%%%%%%%%%%%%%%%%%%%%%%%%%%%%%%%%%%
%%%%%%%%%%%%%%%%%%%%%%%%%%%%%%%%%%%%%%%%%%%%%%%%%%%%%%%%%%%%%%%%%%%%%%%%%%%%
%%%%%%%%%%%%%%%%%%%%%%%%%%%%%%%%%%%%%%%%%%%%%%%%%%%%%%%%%%%%%%%%%%%%%%%%%%%%
%%%%%%%%%%%%%%%%%%%%%%%%%%%%%%%%%%%%%%%%%%%%%%%%%%%%%%%%%%%%%%%%%%%%%%%%%%%%
%%%%%%%%%%%%%%%%%%%%%%%%%%%%%%%%%%%%%%%%%%%%%%%%%%%%%%%%%%%%%%%%%%%%%%%%%%%%
%%%%%%%%%%%%%%%%%%%%%%%%%%%%%%%%%%%%%%%%%%%%%%%%%%%%%%%%%%%%%%%%%%%%%%%%%%%%
%%%%%%%%%%%%%%%%%%%%%%%%%%%%%%%%%%%%%%%%%%%%%%%%%%%%%%%%%%%%%%%%%%%%%%%%%%%%
%%%%%%%%%%%%%%%%%%%%%%%%%%%%%%%%%%%%%%%%%%%%%%%%%%%%%%%%%%%%%%%%%%%%%%%%%%%%
\section{Szemer\'edi's regularity lemma 2}

\end{document}

% \begin{definition}\label{def_coheir_idepencence} 
% For every $a^{\sf s}\in\U^{\sf s*}$ and $b^{\sf s}\in\U^{\sf s*}$ we define
% %
% \ceq{\hfill\emph{$a^{\sf s}\cnonfork_Mb^{\sf s}$}}
% {\IFF}
% {\phi(\U^{\sf s*},b^{\sf s})\cap M^{\sf s}\neq\0
% \textrm{ for all }\phi(x^{\sf s}\,;z^{\sf s})\in L(M,\RR,M^{\sf s}) 
% \textrm{ such that }\phi(a^{\sf s}\,;b^{\sf s})}.


% We define the type

% \ceq{\hfill\emph{$x\cnonfork_Mb$}}
% {=}
% {\Big\{\phi(x\,;{\gr z})
% \ :\ 
% \phi(x\,;b)\in L(M)
% \textrm{ and } M^{|x|}\subseteq\phi(\U\,;b)\Big\}.}

%   % 
% We will use the symbol \emph{$a\equiv_Mx\cnonfork_Mb$} 
% for the union of the types $x\cnonfork_Mb$ and 
% $\tp(a/M)$.\QED
% \end{definition}


\begin{fact}\label{fact_smooth=>fs}
  Let $M$ be given. 
  Every smooth invariant sample is finitely satisfiable and definable.
\end{fact}

\begin{proof}
  Let $s$ be an internal sample that realizes an invariant external sample $p(x^{\sf s})$.
  %
  Let $\phi_i(x^{\sf s}\,;b)\approx_\epsilon\mu_i$ be some finitely many formulas in $p$.
  %
  Let $b'$ be such that $s\cnonfork_{M^{\rm f}}b'\equiv_{M^{\rm f}}b$.
  %
  By invariance, $\sharpphi_i(s\,;b')\approx\mu_i$. 
  %
  Therefore there is an $r\in M^{\rm f}$ such that $\sharpphi_i(r\,;b')\approx_\epsilon\mu_i$.
  %
  As $b'\equiv_{M^{\rm f}}b$, we obtain $\sharpphi_i(r\,;b)\approx_\epsilon\mu_i$.
  %
  This proves finite satisfiability.
  %
  Clearly $\{b:\sharpphi(p\,;b)<\epsilon\}=\{b:\sharpphi(s\,;b)<\epsilon\}$ is definable in $\U^{\rm f*}$. By invariance it is definable in $M^{\rm f}$. 
\end{proof}

\begin{fact}
  Let $M$ be given. Then every external sample that is finitely satisfiable is invariant.
\end{fact}

\begin{proof}
  Let $b\equiv_{M^{\rm f}}b'$ be given.
  %
  Let $r\in M^{\rm f}$ be such that $\sharpphi(p\,;b)\approx_\epsilon\sharpphi(r\,;b)$ and $\sharpphi(p\,;b')\not\approx_\epsilon\sharpphi(r;b')$.
  %
  But $\sharpphi(r\,;b)\approx\sharpphi(r;b')$ and therefore $\sharpphi(p\,;b)\approx_\epsilon\sharpphi(p\,;b')$.
  %
  As $\epsilon$ is arbitrary,  $\sharpphi(p\,;b)\approx\sharpphi(p\,;b')$ follows.
\end{proof}


\begin{fact}
  Let $M$ be given.
  %
  Let $p(x^{\sf s})$ be a smooth sample.
  %
  Then for every $\phi(x\,;z)\in L$ and every $\epsilon$ there are some finitely many $r_i\in M^{\rm f}$, say $0\le i<n$, such that for every $b\in\U^{|z|}$ we have $\phi(p\,;b)\approx_\epsilon\phi(r_i\,;b)$ for some $i$.
\end{fact}

\begin{proof}
  Let $s$ be an internal sample that realizes $p(x^{\sf s})$.
  %
  Let $\phi(x\,;z)\in L$ and $\epsilon$ be fixed.
  %
  The type

  \ceq{\hfill p(z)}
  {=}
  {\Big\{\sharpphi(s\,;z)\not\approx_\epsilon\sharpphi(r\,;z)\ :\ r\in M^{\rm f}\Big\}}
  
  is inconsistent by Fact~\ref{fact_smooth=>fs}.
  %
  Hence compatness yields the required $r_i\in M^{\rm f}$.
\end{proof}


\section{Concrete samples}

In this section we assume that for every $\phi(x),\psi(x)\in L(\U)$ if $\phi(x)\imp\psi(x)$ then $\sharpphi(x^{\sf s})\le\sbar\psi(x^{\sf s})$.


\begin{fact}
  Let $M$ be given.
  %
  Let $p(x^{\sf s})$ be external sample.
  %
  Suppose for every $\phi(x)\in L(\U)$ and $\epsilon$ there are $\theta_1(x), \theta_2(x)\in L(M)$ such that $\theta_1(x)\imp\phi(x)\imp\theta_2(x)$ and $\sbar\theta_2(p)\approx_\epsilon\sbar\theta_1(p)$.
  %
  Then $p$ is finitely satifiable.
\end{fact}

\begin{proof}
  By elementarity there is an $r\in M$ such that $\sbar\theta_2(r)\approx_\epsilon\sbar\theta_1(r)\approx_\epsilon\theta_i(p)$. Then 
\end{proof}





\begin{lemma}
  The following are equivalent for every external sample $p(x^{\sf s})$
  \begin{itemize}
    \item[1.] $p$ is smooth and invariant;
    \item[2.] for every finite $B\subseteq\U^{|z|}$ and every $\epsilon$, there is an $r\in M^{\rm f}$ such that $\sharpphi(p\,;b)\approx_\epsilon\sharpphi(r\,;b)$ for all $b\in B$;
    \item[2.] for every $\phi_i(x\,;z)\in L$, every $b_i\in\U^{|z|}$, and every $\epsilon$, there is an $r\in M^{\rm f}$ such that $\sharpphi_i(p\,;b_i)\approx_\epsilon\sharpphi_i(r\,;b_i)$, for every $0\le i<n$;
    \item[2.] for every $\psi_0(x),\dots,\psi_{n-1}(x)\in L(\U)$ and every $\epsilon$, there is an $r\in M^{\rm f}$ such that $\sbar\psi_i(p)\approx_\epsilon\sbar\psi_i(r)$, for every $i<n$;
    \item[4.] there are finitely many formulas $\theta_i(x)\in L(M)$ such that for every $b\in\U^{|z|}$ we have that $\theta_i(x)\imp\phi(x\,;b)\imp\theta_j(x)$ and $\sbar\theta_j(p)-\sbar\theta_i(p)<\epsilon$ for some $i,j$.
  \end{itemize}
\end{lemma}

\begin{proof} Let $s$ be an internal sample that realizes $p(x^{\sf s})$.

  \ssf{1}$\IMP$\ssf{2}.
  %
  Let $B$ and $\epsilon$ be given.
  %
  Let $\bar b=\<b_i:i<n\>$ be an enumeration of $B$. 
  %
  Let $\mu_i\in\RR$ be such that $\sharpphi(p\,;b_i)\approx\mu_i$ for every $i<n$.
  %
  Let $\bar b'$ be such that $s\cnonfork_{M^{\rm f}}\bar b'\equiv_{M^{\rm f}}\bar b$.
  %
  By invariance, $\sharpphi(s\,;b'_i)\approx\mu_i$. 
  %
  Therefore there is an $r\in M^{\rm f}$ such that $\sharpphi(r\,;b'_i)\approx_\epsilon\mu_i$ for every $i<n$.
  %
  As $\bar b'\equiv_{M^{\rm f}}\bar b$, we obtain  $\sharpphi(r\,;b_i)\approx_\epsilon\mu_i$.


  \ssf{1}$\IMP$\ssf{2}.
  Let $\psi_i(x)=\phi_i(x\,;b_i)$ for some $\phi_i(x\,;z)\in L$ and  $b_i\in\U^{|z|}$.
  %
  Let $\bar b=\<b_i:i<n\>$. 
  %
  Let $\mu_i\in\RR$ be such that $\sharpphi_i(p\,;b_i)\approx\mu_i$.
  %
  Let $\bar b'$ be such that $s\cnonfork_{M^{\rm f}}\bar b'\equiv_{M^{\rm f}}\bar b$.
  %
  By invariance, $\sharpphi_i(s\,;b'_i)\approx\mu_i$. 
  %
  Therefore there is an $r\in M^{\rm f}$ such that $\sharpphi_i(r\,;b'_i)\approx_\epsilon\mu_i$ for every $i<n$.
  %
  As $\bar b'\equiv_{M^{\rm f}}\bar b$, we obtain  $\sharpphi_i(r\,;b_i)\approx_\epsilon\mu_i$.


  \ssf{1}$\IMP$\ssf{3}.
  %
  By smoothness we it suffices to prove \ssf{3} with $s$ for $p$.
  %
  The type 

  \ceq{\hfill p(z)}{=}{\Big\{\sharpphi(s\,;z)\not\approx_\epsilon\sharpphi(r\,;z)\ :\ r\in M^{\rm f}\Big\}}
  
  is inconsistent by \ssf{2}.
  %
  Hence compatness yields the required $r_1,\dots,r_n\in M^{\rm f}$.

  \ssf{3}$\IMP$\ssf{2}. Clear.

  \ssf{2}$\IMP$\ssf{1} Assume \ssf{2}.
  %
  We prove that $p$ is invariant.
  %
  Pick some $b\equiv_{M^{\rm f}}b'$ and some $\epsilon$.
  %
  Let $r,r'\in M^{\rm f}$ be such that $\sharpphi(p\,;b)\approx_\epsilon\sharpphi(r\,;b)$ and $\sharpphi(p\,;b')\not\approx_\epsilon\sharpphi(r';b')$.
  %
  As $\sharpphi(r\,;b)\approx\sharpphi(r';b')$ follows from  $b\equiv_{M^{\rm f}}b'$, we obtain $\sharpphi(p\,;b)\approx_\epsilon\sharpphi(p\,;b')$.

  We prove that $p$ is smooth. 
  
  \ceq{\hfill p(x^{\sf s})}{=}{\Big\{\sharpphi(x^{\sf s}\,;b)\approx_\epsilon\mu_b\ :\ b\in\U^{|z|}\Big\}}
\end{proof}

%Clearly, if $s$ is an internal sample, then $p(x^{\sf s})=\tp(s/\U^{\rm f})$ is a smooth sample w.r.t.\@ any given formula $\phi(x\,;z)\in L$.

\begin{definition}\label{def_genstable}
  Let $\phi(x\,;z)\in L$ and $M$ be given.
  %
  A external sample $p(x^{\sf s})$ is
  \begin{itemize}
    \item[1.]\noindent \emph{invariant\/} if $\sharpphi(p\,;b)\approx\sharpphi(p\,;b')$ for every $b\equiv_{M^{\rm f}}b'$;
    \item[2.]\noindent (pointwise) \emph{approximable\/} if, for every $b\in\U^{|z|}$ and every $\epsilon$, there is an $r\in M^{\rm f}$ such that $\sharpphi(p\,;b)\approx_\epsilon\sharpphi(r\,;b)$;
    \item[3.]\noindent \emph{definable\/} if, for every $\epsilon$, the set $\{b\in\U^{|z|}:\sharpphi(p\,;b)<\epsilon\}$ is definable over $M^{\rm f}$.\QED 
  \end{itemize}
\end{definition}

The following is a useful characterization of smooth samples.

\section{Old stuff/garbage}

 
 %
 We write \emph{$\supp(p)$\/} for the \emph{support\/} of $p$, that is, the set of those $a$ in $\U$ such that $p\proves (a\in x^{\sf s})>\epsilon$ for some standard positive $\epsilon$.
 %
 If $s$ is a smooth sample  \emph{$\supp(s)$\/} is defined to be  \emph{$\supp(p)$\/} for $p(x^{\sf s})=\tp(s/\U^{\rm f})$. 
 


% \begin{definition}[???]
%   We say that $\U$ is \emph{pseudofinite\/} every definable set is the support of a sample.\QED
% \end{definition}

Let $s$ be an external sample.
%
For every formula $\phi(x\,;z)\in L$ and every $b\in\U^{|z|}$ we define

\ceq{\hfill\emph{$\Av_{x/s}\phi(x\,;b)$} }
{=}
{\frac{\big|\big\{a\in s\ :\ \phi(a\,;b)\big\}\big|}
{|s|}},

When possible we abbreviate $\Av_{x/s}\phi(x\,;b)$ with \emph{$\sharpphi(s\,;b)$}.
%
%If we pretend that hyperrational are elements of $\U^{\rm f^+_*}$, then $\sharpphi(s\,;b)$, as a function of $s$ and $b$, is definable in $\U^{\rm f^+_*}$.

%
We write $\st(a)$, for the standard part of the hyperrational number $a$.
%
That is, the unique real number $\mu$ such that $\mu\approx a$.

The standard part of $\Av_s$ induces a finite probability measure on the algebra of definable subsets of $\U^{|x|}$.
%
This measure has an unique extension to a Lebesgue probability measure on a $\sigma$-algebra.
%
This is known as the \emph{Loeb measure.}


All definitions and facts in this section are relative to some given formula $\phi(x\,;z)\in L$ and some model $M$. 

\begin{definition} Let $\phi(x\,;z)\in L$ and $M$ be given. We say that a sample $s$ is \emph{smooth\/} if, for every $s'\equiv_Ms$ and every $b\in\U^{|z|}$, we have $\sharpphi(s';b)\approx\sharpphi(s\,;b)$.\QED
\end{definition}

\begin{definition}\label{def_genstable}
  Let $\phi(x\,;z)\in L$ and $M$ be given.
  %
 An external sample $s$ is 
  \begin{itemize}
    \item[1.]\noindent \emph{invariant\/} if, for every $b,b'\in\U^{|z|}$ such that $b\equiv_Mb'$, we have $\sharpphi(s\,;b)\approx\sharpphi(s\,;b')$;
    \item[2.] \noindent \emph{definable\/} if, for every $\epsilon$, the set $\{b\in\U^{|z|}:\sharpphi(s\,;b)<\epsilon\}$ is definable over $M$;
    % \item [3.]\noindent  \emph{satisfiable\/} if, for all $b\in\U^{|z|}$, if $\sharpphi(s\,;b)>\epsilon$, then  $\phi(r\,;b)>\epsilon$ for some $r\in M^{\rm f}$.
    \item [3.]\noindent  \emph{finitely satisfiable\/} if, for all $b\in\U^{|z|}$ there is an $r\in M^{\rm f}$ such that  $\sharpphi(s\,;b)\approx_\epsilon\sharpphi(r\,;b)$;
    \item[4.] \noindent  \emph{generically stable\/} if all of the above hold.\QED 
  \end{itemize}
  %When $\phi(x\,;z)$ and $M$ are not clear from the context we add: \emph{for $\phi(x\,;z)$ over/in $M$.}
  \end{definition}

  Smooth roughly means internal and invariant.

  \begin{fact}\label{fact_smoothinternalinvariant}
    Let $\phi(x\,;z)\in L$ and $M$ be given.
    %
    The following are equivalent for every external sample $s$

    \begin{itemize}
      \item[1.] $s$ is smooth;
      \item[2.] there is an invariant internal sample $s'\equiv_Ms$.
    \end{itemize}
  \end{fact}

\begin{proof}
  \ssf{1}$\IMP$\ssf{2}.
  %
  Let $s$ be smooth and let $s'\equiv_Ms$ be any internal sample.
  %
  If $b,b'\in\U^{|z|}$ and $b\equiv_Mb'$, then $b'=fb$ for some $f\in\Aut(\U^{\rm f*}/M)$.
  %
  Then $\sharpphi(s';fb)\approx\sharpphi(f^{-1}s'\,;b)$.
  %
  Moreover, by smothness $\sharpphi(f^{-1}s'\,;b)\approx\sharpphi(s';b)$. The infariace in $s'$ follows.
  
  \ssf{2}$\IMP$\ssf{1}.
  %
  It suffices to prove that if $s$ is internal and invariant then it is smooth.
  %
  Suppose for a contradiction that there is an $s'\equiv_Ms$ such that $\sharpphi(s'\,;b)\not\approx_\epsilon\sharpphi(s\,;b)$.
  %
  Pick a internal sample $s''\equiv_{M,s}s'$. 
  %
  Then $\sharpphi(s''\,;b)\not\approx_\epsilon\sharpphi(s\,;b)$.
  %
  As $s''\equiv_Ms$ are both in $\U^{\rm f*}$, then $s''=fs$ for some $f\in\Aut(\U^{\rm f*}/M)$.
  %
  Hence we obtan $\sharpphi(s\,;f^{-1}b)\not\approx_\epsilon\sharpphi(s\,;b)$ which contradics the invariance of $s$.
\end{proof}


\begin{definition}\label{def_approximable_sample}
  Let $\phi(x\,;z)\in L$ and $M$ be given.
  %
  Let $\B\subseteq\U^{|z|}$ be arbitrary.
  %
  We say that $s$ is \emph{(uniformly) approximable\/} on $\B$ if for every $\epsilon$ there is a finite sample $r\in M^{\rm f}$ such that $\sharpphi(s,b)\approx_\epsilon\sharpphi(r,b)$ for every $b\in\B$. (The sample $r$ depends on $\epsilon$, not on $b$.)\QED
\end{definition}

\begin{lemma}\label{lem_cons}
  Let $\phi(x\,;z)\in L$ and $M$ be given.
  %
  Let $s$ be approximable on $\B$ and such that $\sharpphi(s\,;b)>\epsilon$ for every $b\in\B$. Then there is a finite cover of $\B$, say $\B_1,\dots,\B_n$, such that all the types $p_i(x)=\{\phi(x\,;b):b\in\B_i\}$ are consistent.
\end{lemma}

\begin{proof}
  Let $r\in M^{\rm f}$ be as in Definition~\ref{def_approximable_sample}.
  %
  As $r$ is finite, we may assume that $\supp(r)=\{a_1,\dots,a_n\}$.
  %
  Let $\B_i=\{b\in\B:\phi(a_i\,;b)\}$.
  % 
  As $\sharpphi(r\,;b)>0$ for all $b\in\B$, these $\B_i$ are the required cover of $\B$.
\end{proof}

\begin{corollary}
  Let $\phi(x\,;z)\in L$ and $M$ be given.
  %
  If $s$ be approximable on $\<b_i:i<\omega\>$, where $\<b_i:i<\omega\>$ is a sequence of indiscernibles such that $\sharpphi(s\,;b_i)>\epsilon$ for every $i<\omega$, then the type $\{\phi(x\,;b_i):i<\omega\}$ is consistent.\QED
\end{corollary}

\begin{corollary}
  Let $\phi(x\,;z)\in L$ and $M$ be given.
  %
  Let $s$ be approximable on $\C$.
  %
  Then for every $\epsilon$ there is a pair of distict $c,c'\in\C$ such that $\Av_{x/s}\big[\phi(x\,;c)\niff\phi(x\,;c')\big]<\epsilon$
\end{corollary}

\begin{proof}
  Suppose for a contradiction that $\Av_{x/s}\big[\phi(x\,;c)\niff\phi(x\,;c')\big]\ge\epsilon$ for all distict $c,c'\in\C$. Apply Lemma~\ref{lem_cons} to the folmula $\psi(x\,;z,z') = [\phi(x\,;z)\niff\phi(x\,;z')]$ and the set $\B=\{\<c,c'\>\in\C^2\, :\, c{\neq} c'\}$.
  %
 The sets $\B_i$ obtained from Lemma~\ref{lem_cons} induce a finite coloring of the complete graph on $\C$.
 %
 By the Ramsey theorem there is an infinite monochromatic set $A\subseteq\C$.
 %
 Hence $\{\phi(x\,;a)\niff\phi(x\,;a')\  :\  a,a'{\in} A,\ a{\neq} a'\}$ is consistent.
 %
 As $|A|>2$, this is impossible.
\end{proof}


% The following is stronger form of invariance.


% \begin{definition}
%   Let  $\phi(x\,;z)\in L$.
  
%   We say that a sample $s$ of $M^{|x|}$ is $\phi$-smooth over $M$ if there is a finite $S\subseteq M^{|x|}$ such that $\sharpphi(s\,;b)\approx\sharpphi(s\,;b)$ for all $b\in M^{|z|}$.
%   We say that an external sample $s$ is \emph{smooth\/} for $\phi(x\,;z)$ over $M$ if for any other sample $s'$ such that $\sharpphi(s\,;b)\approx\sharpphi(s';b)$ for all $b\in M^{|z|}$, the same equivalence holds for all $b\in\U^{|z|}$.\QED
% \end{definition}

% Clearly, smooth impies invariant.


%%%%%%%%%%%%%%%%%%%%%%%%%%%%%%%%%%%%%%%%%%%%%%%%%%%%%%%%%%%%%%%%%%%%%%%%%%%%
%%%%%%%%%%%%%%%%%%%%%%%%%%%%%%%%%%%%%%%%%%%%%%%%%%%%%%%%%%%%%%%%%%%%%%%%%%%%
%%%%%%%%%%%%%%%%%%%%%%%%%%%%%%%%%%%%%%%%%%%%%%%%%%%%%%%%%%%%%%%%%%%%%%%%%%%%
%%%%%%%%%%%%%%%%%%%%%%%%%%%%%%%%%%%%%%%%%%%%%%%%%%%%%%%%%%%%%%%%%%%%%%%%%%%%
%%%%%%%%%%%%%%%%%%%%%%%%%%%%%%%%%%%%%%%%%%%%%%%%%%%%%%%%%%%%%%%%%%%%%%%%%%%%
%%%%%%%%%%%%%%%%%%%%%%%%%%%%%%%%%%%%%%%%%%%%%%%%%%%%%%%%%%%%%%%%%%%%%%%%%%%%
%%%%%%%%%%%%%%%%%%%%%%%%%%%%%%%%%%%%%%%%%%%%%%%%%%%%%%%%%%%%%%%%%%%%%%%%%%%%
%%%%%%%%%%%%%%%%%%%%%%%%%%%%%%%%%%%%%%%%%%%%%%%%%%%%%%%%%%%%%%%%%%%%%%%%%%%%
\section{The nip formulas}\label{nip}

\begin{theorem}\label{thm_VC}
  Let $\phi(x\,;z)\in L$ and $M$ be given and assume that $\phi(x\,;z)$ is nip.
  %
  Then every sample is approximable on $M$ over $M$.
\end{theorem}

\begin{proof}
  Questo è Vapnik-Chervonenkis.
\end{proof}

For smooth samples the Vapnik-Chervonenkis Theorem can be strengthened as follows. (The difference is that below $b$ ranges over $\U^{|z|}$.)

\begin{corollary}
  Let  $\phi(x\,;z)\in L$ be nip.
  %
  Every smooth sample $s$ is approximable on $\U$ over $M$.
\end{corollary}

\begin{proof}
  By Theorem~\ref{thm_VC}, there is an $r\in M^{\rm f}$ such that $\sharpphi(s\,;b)\approx_\epsilon\sharpphi(r\,;b)$ for all $b\in M^{|z|}$.
  
  Suppose for a contradiction that $\sharpphi(s\,;b')\not\approx_\epsilon\sharpphi(r\,;b')$ for some $b'\in\U^{|z|}$.
  %
  Pick a  $s'\subseteq\U^{|x|}$ such that $b'\cnonfork_{M}s'\equiv_Ms$.
  %
  By the smoothness of $s$, we obtain $\sharpphi(s';b')\not\approx_\epsilon\sharpphi(r\,;b')$.
  %    
  As $b'\cnonfork_{M}s'$, the same formula holds for some $b\in M^{|z|}$.
  %
  Once again by smoothness, $\sharpphi(s\,;b)\not\approx_\epsilon\sharpphi(r\,;b)$.
  %
  A contradiction.
\end{proof}

\begin{definition}[??]
  We say that $\phi(x\,;z)\in L$ is \emph{distal\/} if there is a formula $\psi(x\,;z_1,\dots z_n)\in L$ such that for every finite set $B\subseteq\U$ and every $a\in\U^{|x|}$ there are $b_1,\dots,b_n\in B$ such that $\psi(a\,;b_1,\dots,b_n)$ and $\psi(x\,;b_1,\dots,b_n)$ decides all formulas $\phi(x\,; b)$ for $b\in B$.\QED
\end{definition}

\begin{definition}[??*]
  We say that $\phi(x\,;z)\in L$ is \emph{distal\/} if there is a formula $\psi(x_1,\dots,x_n\,;z)\in L$ such that for every finite set $A\subseteq\U^{|x|}$ and every $b\in\U^{|z|}$ there are $a_1,\dots,a_n\in B$ such that $\psi(a_1,\dots,a_n,\;z)$ and $\psi(a_1,\dots,a_n,\,z)$ decides all formulas $\phi(a\,; z)$ for $a\in A$.\QED
\end{definition}

For every finite sample $r\in M^{\rm f}$ and every $\mu\in\RR$ there is a formula $\psi(z)\in L(\supp r)$ such that $\sharpphi(r\,;z)=\mu \iff \psi(z)$. The formula $\psi(z)$ depends on $r$.

When $\phi(x\,;z)$ is distal 


If  $\phi(x\,;z)$ is distal then there is a formula $\psi$ such that  for every $r\in M^f$ there is $r_0\in M^{}$

\begin{theorem}[(false)]
  The following are equivalent
  \begin{itemize}
    \item[1.] $\phi(x\,;z)\in L$ is distal;
    \item[2.] for every sample $s\in\U^{\rm f^+_*}$, if $s$ is generically stable then $s$ is smooth.

  \end{itemize}
\end{theorem}

There is a formula $\psi(x_1,\dots,x_n\,;z)$ such that for every $r\in M^{\rm f}$ 

there are $a_1,\dots,a_n$ such that $\phi(r\,;b)=\mu$ $\psi(x_1,\dots,x_n\,;z)$


\end{document}
%%%%%%%%%%%%%%%%%%%%%%%%%%%%%%%%%%%%%%%%%%%%%%%%%%%%%%%%%%%%%%%%%%%%%%%%%%%%
%%%%%%%%%%%%%%%%%%%%%%%%%%%%%%%%%%%%%%%%%%%%%%%%%%%%%%%%%%%%%%%%%%%%%%%%%%%%
%%%%%%%%%%%%%%%%%%%%%%%%%%%%%%%%%%%%%%%%%%%%%%%%%%%%%%%%%%%%%%%%%%%%%%%%%%%%
%%%%%%%%%%%%%%%%%%%%%%%%%%%%%%%%%%%%%%%%%%%%%%%%%%%%%%%%%%%%%%%%%%%%%%%%%%%%
%%%%%%%%%%%%%%%%%%%%%%%%%%%%%%%%%%%%%%%%%%%%%%%%%%%%%%%%%%%%%%%%%%%%%%%%%%%%
%%%%%%%%%%%%%%%%%%%%%%%%%%%%%%%%%%%%%%%%%%%%%%%%%%%%%%%%%%%%%%%%%%%%%%%%%%%%
%%%%%%%%%%%%%%%%%%%%%%%%%%%%%%%%%%%%%%%%%%%%%%%%%%%%%%%%%%%%%%%%%%%%%%%%%%%%
%%%%%%%%%%%%%%%%%%%%%%%%%%%%%%%%%%%%%%%%%%%%%%%%%%%%%%%%%%%%%%%%%%%%%%%%%%%%
\section{Uniform samples}\label{samples}
We call $\U$ the domain of the home sort.
%
This curries all the structure.
%
The domain of the second sort, 

Fix a tuple of variables $x$ and let $\bar x=\<x_i:i<\omega\>$, where $|x_i|=|x|$.
%
For $p(\bar x)\in S(\U)$ and $\psi(x)\in L(\U)$ we define ($n>1$)

\ceq{\hfill\emph{$\Av_{p\restriction n}\,\psi(x)$} }
{=}
{\frac1n\Big|\big\{i<n\ :\ p(\bar x)\proves\psi(x_i)\big\}\Big|}

If $p(\bar x)$ is the type containing $\b ar x=\bar a$ for some $\bar a\in \U^{|x|\cdot\omega}$,
we write \emph{$\Av_{\bar a{\restriction n}}\,\psi(x)$.}

\begin{definition}
  We say that $p(\bar x)\in S(\U)$ is a \emph{sample\/}
  if the limit below exists for every formula $\psi(x)\in L(\U)$.

  \ceq{\hfill\emph{$\Av_p\,\psi(x)$}}{=}{\lim_{n\to\infty}\Av_{p\restriction n}\,\psi(x).}

  Let $\phi(x,z)\in L$ be given.
  %
  We say that $p(\bar x)$ is a \emph{$\phi$-sample\/} 
  if the formula $\psi(x)$ above is restricted to range over those of the form $\phi(x,b)$ 
  for $b\in\U^{|z|}$.

  We say that $p(\bar x)$ is a \emph{uniform $\phi$-sample\/} if it is a sample and 
  for every $\epsilon>0$ there is a $k$ such that $\Av_{p\restriction n}\,\phi(x,b)$ 
  is within $\epsilon$ from $\Av_p\,\phi(x,b)$ for all $n>k$ and $b\in\U^{|z|}$.

  We say that $p(\bar x)$ is a \emph{uniform sample\/} if it is a uniform 
  $\phi$-sample for all $\phi(x,z)\in L$.\QED
\end{definition}

\begin{proposition}
  Let $p(\bar x)\in S(\U)$ be a sample finitely satisfied in $M$.
  %
  Let $q(x)\in S(M)$.
  %
  Then for every formula $\phi(x)\in L(\U)$ either 
  
  \hfil$\displaystyle\inf\Big\{\Av_p\,\big(\psi(x)\wedge\phi(x)\big)\ :\ \psi(x)\in q\Big\}$
  
  is either $0$ or $\Av_p\,\phi(x)$.
\end{proposition}
\begin{proof}
  As $p(\bar x)$ is finitely satisfied in $M$ there is a tuple in $\bar a\in M^{|x|\cdot n}$
  such that $\Av_{\bar a}\,\big(\psi(x)\wedge\phi(x)\big)$ is within $\epsilon$ form the 
  infimum above.
\end{proof}

Define

\ceq{\hfill\emph{$\Av_{p\restriction n}\,q(x)$}}
{=}
{\inf\Big\{\Av_{p\restriction n}\,\psi(x)\ :\ \psi(x)\in q\Big\}}

Clearly the infimum above is attained.

\begin{proposition}
      If $p(\bar x)$ is a sample and $q(x)\subseteq L(\U)$, then
      
      \ceq{\hfill\lim_{n\to\infty}\Av_{p\restriction n}\,q(x)}
      {=}
      {\inf\Big\{\Av_p\,\phi(x)\ :\ \phi(x)\in q\Big\}}
\end{proposition}

\begin{proof}
      Fix $\epsilon>0$.
      %
      Let $\phi(x)\in q$ be such that $\Av_p\,\phi(x)$ is within $\epsilon$ from the infimum above.
      %
      Let $k$ be such that $\Av_{p\restriction n}\,\phi(x)$ is within $\epsilon$ from  $\Av_p\,\phi(x)$ for every $n>k$.
      %
      Then, for every $n>k$,

      \ceq{\hfill\Av_{p\restriction n}\,q(x)}
      {\le}
      {\Av_{p\restriction n}\,\phi(x)}

      \ceq{}
      {\le}
      {\Av_p\,\phi(x) + \epsilon}

      \ceq{}
      {\le}
      {\inf\Big\{\Av_p\,\phi(x)\ :\ \phi(x)\in q\Big\} + 2\epsilon}

      For the converse inequality we inductively pick a sequence of integers $n_i$ and formulas $\psi_i(x)\in q$ as follows.
      %
      Let $n_0=1$ and $\psi_i(x)$ such that $\Av_{p\restriction {n_i}}\,\psi_i(x)=\Av_{p\restriction {n_i}}\,q(x)$.
      %
      Then $n_{i+1}$ be such that $\Av_p\,\psi_i(x)\le\Av_{p\restriction {n_{i+1}}}\,\psi_i(x)+\epsilon$.
      %
      Then

      \ceq{\hfill\inf\Big\{\Av_p\,\phi(x)\ :\ \phi(x)\in q\Big\}}
      {\le}
      {\inf\Big\{\Av_p\,\psi_i(x)\ :\ i<\omega\Big\}}

      \ceq{}
      {\le}
      {\inf\Big\{\Av_{p\restriction {n_{i+1}}}\,\psi_i(x)\ :\ i<\omega\Big\} + \epsilon}
      % {\Av_{p\restriction n}\,q(x) + \epsilon}
\end{proof}

\begin{corollary}
      Let $p(\bar x)$ be a sample finitely satisfied in $M$.
      %
      Let $q'(x)=q(x)\cup\{\phi(x)\}$ for $q(x)\in S(M)$ and $\phi(x)\in L(\U)$.
      %
      Then $\Av_p\,q'(x)$ is either $0$ or $1$.
\end{corollary}
\begin{proof}
Under the assumptions of the corollary, for every $\epsilon>0$ there is an $n$ and a tuple $\bar a\in M^{|x|\cdot\omega}$ such that $\Av_{\bar a{\restriction n}}\,q'(x)$ is within $\epsilon$ from $\Av_p\,q'(x)$. As $\Av_{\bar a{\restriction n}}\,q'(x)$
\end{proof}



\begin{proposition}
      Let $\bar x=\<x_i:i<\omega\>$ and $|x_i|=|x|$ and let $p(\bar x)\in S(\U)$.
      %  
      Then the following are equivalent 
      \begin{itemize}
            \item[1.] $p(\bar x)$ is a uniform external sample;
            \item[2.] for every $\epsilon>0$, there is an $n$ and a formula $\theta(\bar x)\in p$ such that $\Av_n\big(\bar a;\phi(x,b)\big)$ is within $\epsilon$ from $\Av_p\phi(x,b)$ for all $b\in\U^{|z|}$ and all $\bar a\models\theta(\bar x)$.
      \end{itemize}
\end{proposition}

\begin{proof}
???
\end{proof}

Vale anche/solo/nemmeno la versione non uniforme della proposizione?

Note that, if $p(\bar x)$ is definable, say over $M$, then there is a formula $\psi_{m/n}(z)\in L(M)$ such that

\ceq{\hfill\psi_{m/n}(z)}{\IFF}{\Av_n\big(p(\bar x);\phi(x,z)\big)=\frac{m}{n}.}



%%%%%%%%%%%%%%%%%%%%%%%%%%%%%%%%%%%%%%%%%%%%%%%%%%%%%%%%%%%%%%%%%%%%%%%%%%%%
%%%%%%%%%%%%%%%%%%%%%%%%%%%%%%%%%%%%%%%%%%%%%%%%%%%%%%%%%%%%%%%%%%%%%%%%%%%%
%%%%%%%%%%%%%%%%%%%%%%%%%%%%%%%%%%%%%%%%%%%%%%%%%%%%%%%%%%%%%%%%%%%%%%%%%%%%
%%%%%%%%%%%%%%%%%%%%%%%%%%%%%%%%%%%%%%%%%%%%%%%%%%%%%%%%%%%%%%%%%%%%%%%%%%%%
%%%%%%%%%%%%%%%%%%%%%%%%%%%%%%%%%%%%%%%%%%%%%%%%%%%%%%%%%%%%%%%%%%%%%%%%%%%%
%%%%%%%%%%%%%%%%%%%%%%%%%%%%%%%%%%%%%%%%%%%%%%%%%%%%%%%%%%%%%%%%%%%%%%%%%%%%
%%%%%%%%%%%%%%%%%%%%%%%%%%%%%%%%%%%%%%%%%%%%%%%%%%%%%%%%%%%%%%%%%%%%%%%%%%%%
\section{References}
\begin{biblist}[]\normalsize

\bib{DK}{book}{
   author={Dodos, Pandelis},
   author={Kanellopoulos, Vassilis},
   title={\href{http://users.uoa.gr/~pdodos/Publications/RT.pdf}
         {Ramsey theory for product spaces}},
   series={Mathematical Surveys and Monographs},
   volume={212},
   publisher={American Mathematical Society},
%   publisher={American Mathematical Society, Providence, RI},
   date={2016},
}
\bib{DZ}{book}{
   author={Zambella, Domenico},
   title={\href{https://github.com/domenicozambella/creche/raw/master/creche.pdf}
         {A cr\`eche course in model theory}},
   date={2018}, 
   series={AMS Open Math Notes}, 
   note={(The link points to the github version)}
}
\end{biblist}

\end{document}
