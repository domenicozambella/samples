\documentclass[10pt,openany]{article}
\usepackage[utf8]{inputenc}
\usepackage[a4paper,hmargin={4cm,4cm},vmargin={3.5cm,3.5cm}]{geometry}
\usepackage[colorlinks=true,bookmarksopen=false,linkcolor=blue,citecolor=red]{hyperref}
\usepackage{subfiles}
\usepackage{calc} 
\usepackage{datetime}
\usepackage{comment}
\usepackage{amssymb}
\usepackage{amsthm}
\usepackage{amsmath}
\usepackage{amsrefs}
\usepackage{titlesec}
\usepackage{titletoc}
\usepackage{dsfont}
\usepackage{euscript}
\usepackage{fourier-orns}
%\usepackage{pxfonts}
%\usepackage{newpxtext}
%\usepackage{tgpagella}
\usepackage{palatino}
\usepackage[sc]{mathpazo} % add possibly `sc` and `osf` options
%\usepackage{eulervm}
%\usepackage[adobe-utopia]{mathdesign}
\usepackage[T1]{fontenc}
\usepackage{graphicx}
\usepackage{tikz}
% \usetikzlibrary{tikzmark}
% \usepackage{tikz-cd}
% \tikzcdset{
% arrow style=tikz,
% diagrams={>=latex}
% }

\linespread{1.265}
\setlength{\parindent}{0ex}
\setlength{\parskip}{.4\baselineskip}
\definecolor{blue}{rgb}{0, 0.1, 0.6}

\DeclareFontFamily{OT1}{pzc}{}
\DeclareFontShape{OT1}{pzc}{m}{it}{<-> s * [1.10] pzcmi7t}{}
\DeclareMathAlphabet{\mathpzc}{OT1}{pzc}{m}{it}

\newcommand{\mylabel}[1]{{\ssf{#1}}\hfill}
\renewenvironment{itemize}
  {\begin{list}{$\triangleright$}{%
   \setlength{\parskip}{0mm}
   \setlength{\topsep}{.4\baselineskip}
   \setlength{\rightmargin}{0mm}
   \setlength{\listparindent}{0mm}
   \setlength{\itemindent}{0mm}
   \setlength{\labelwidth}{3ex}
   \setlength{\itemsep}{.4\baselineskip}
   \setlength{\parsep}{0mm}
   \setlength{\partopsep}{0mm}
   \setlength{\labelsep}{1ex}
   \setlength{\leftmargin}{\labelwidth+\labelsep}
   \let\makelabel\mylabel}}{%
   \end{list}\vspace*{-\parskip}}

\def\E{\exists}
\def\A{\forall}
\def\mdot{\mathord\cdot}
\def\models{\vDash}
\def\pmodels{\mathrel{\models\kern-1.5ex\raisebox{.5ex}{*}}}
\def\notmodels{\nvDash}
\def\proves{\vdash}
\def\notproves{\nvdash}
\def\proves{\vdash}
\def\provesT{\mathrel{\mathord{\vdash}\hskip-1.13ex\raisebox{-.5ex}{\tiny$T$}}}
\def\ZZ{\mathds Z}
\def\NN{\mathds N}
\def\QQ{\mathds Q}
\def\RR{\mathds R}
\def\BB{\mathds B}
\def\CC{\mathds C}
\def\PP{\mathds P}
\def\Ar{{\rm Ar}}
\def\dom{\mathop{\rm dom}}
\def\supp{\mathop{\rm supp}}
\def\range{\mathop{\rm img}}
\def\rank{\mathop{\rm rank}}
\def\dcl{\mathop{\rm dcl}}
\def\acl{\mathop{\rm acl}}
\def\fp{{\rm fp}}
\def\cf{\mathop{\rm cf}}
\def\rad{\mathop{\rm rad}}
\def\eq{{{\rm eq}}}
\def\ccl{{\rm ccl}}
\def\Th{\textrm{Th}}
\def\Diag{{\rm Diag}}
\def\atpmTh{\textrm{Th}_\atpm}
\def\Mod{\mathop{\rm Mod}}
\def\Rmod{{\mbox{\scriptsize $R$-mod}}}
\def\Aut{\textrm{Aut\kern.15ex}}
\def\Autf{\mathord{\rm Aut\kern.15ex{f}\kern.15ex}}
\def\Cb{\textrm{Cb\kern.15ex}}
\def\orbit{\O}
\def\oorbit{\mathpzc{o}}
\def\oorbitf{\mathpzc{of\!}}
\def\id{{\rm id}}
\def\tp{{\rm tp}}
\def\atpm{{\tiny\rm at^\pm}}
\def\qftp{\textrm{qf\mbox{-}tp}}
\def\attp{\textrm{at\mbox{-}tp}}
\def\atpmtp{\atpm\mbox{-}\textrm{tp}}
\def\Deltatp{\Delta\mbox{-}\textrm{tp}}
\def\pmDelta{\Delta\hskip-.3ex\raisebox{1ex}{\tiny$\pm$}}
\def\pmDeltatp{\noindent\pmDelta\hskip-.3ex\textrm{-tp}}
\def\EMtp{\textrm{{\small EM}\mbox{-}tp}}

\newcommand{\cev}[1]{\reflectbox{\ensuremath{\vec{\reflectbox{\ensuremath{#1}}}}}}

\def\nonfork{\mathop{\raise0.2ex\hbox{
   \ooalign{\hidewidth$\vert$\hidewidth\cr\raise-0.9ex\hbox{$\smile$}}}}}

\def\cnonfork{\mathbin{\raise1.8ex\rlap{\kern0.6ex\rule{0.6ex}{0.1ex}}
\rlap{\kern1.1ex\rule{0.1ex}{1.9ex}}\raise-0.3ex\hbox{$\smile$} } }

\def\cpaw{\mathbin{\ooalign{\kern-0.4ex$-$\hidewidth\cr$<$}}}
\def\cpawdot{\ooalign{$\kern1.2ex\cdot$\cr$\cpaw$\cr}}

\def\sm{\smallsetminus}
\def\Trm{{\rm Trm}}
\def\IMP{\Rightarrow}
\def\PMI{\Leftarrow}
\def\IFF{\Leftrightarrow}
\def\NIFF{\nLeftrightarrow}
\def\imp{\rightarrow}
\def\pmi{\leftarrow}
\def\iff{\leftrightarrow}
\def\niff{\mathrel{{\leftrightarrow}\llap{\raisebox{-.1ex}{{\small$/$}}\hskip.5ex}}}
\def\nequiv{\mathrel{\mbox{$\equiv$\llap{{\small$/$}\hskip.3ex}}}}

\def\isomap{\rlap{\kern0.8ex\raisebox{1ex}{\scriptsize$\sim$}}\rightarrow}

\def\P{\EuScript P}
\def\D{\EuScript D}
\def\Aa{\EuScript A}
\def\Ee{\EuScript E}
\def\X{\EuScript X}
\def\Y{\EuScript Y}
\def\Z{\EuScript Z}
\def\C{\EuScript C}
\def\U{\EuScript U}
\def\W{\EuScript W}
\def\Hh{\EuScript H}
\def\I{\EuScript I}
\def\V{\EuScript V}
\def\R{\EuScript R}
\def\F{\EuScript F}
\def\G{\EuScript G}
\def\B{\EuScript B}
\def\M{\EuScript M}
\def\N{\EuScript N}
\def\Ll{\EuScript L}
\def\K{\EuScript K}
\def\O{\EuScript O}
\def\J{\EuScript J}
\def\S{\EuScript S}
\def\T{\EuScript T}
\def\<{\langle}
\def\>{\rangle}
\def\0{\varnothing}
\def\theta{\vartheta}
\def\phi{\varphi}
\def\epsilon{\varepsilon}
\def\ssf#1{\textsf{\small #1}}

\titlecontents{section}
[3.8em] % ie, 1.5em (chapter) + 2.3em
{\vskip-1ex}
{\contentslabel{1.5em}}
{\hspace*{-2.3em}}
{\titlerule*[1pc]{}\contentspage}

\titleformat{\section}[block]{\Large\bfseries}{\makebox[5ex][r]{\textbf{\thesection}}}{1.5ex}{}
\titlespacing*{\chapter}{0em}{.5ex plus .2ex minus .2ex}{2.3ex plus .2ex}
\titlespacing*{\section}{-9.7ex}{3ex plus .5ex minus .5ex}{1ex plus .2ex minus .2ex}

\newtheoremstyle{mio}% name
     {2\parskip}%      Space above
     {\parskip}%      Space below
     {\sl}%         Body font
     {}%          Indent amount (empty = no indent, \parindent = para indent)
     {\bfseries}% Thm head font
     {}%        Punctuation after thm head
     {1ex}%     Space after thm head: " " = normal interword space;
         %   \newline = linebreak
     {\llap{\thmnumber{#2}\hskip2mm}% Thm head spec (empty means `normal')
      \thmname{#1}\thmnote{\bfseries{} #3}}

\newtheoremstyle{liscio}% name
     {2\parskip}%      Space above
     {0mm}%      Space below
     {}%         Body font
     {}%         Indent amount (empty = no indent, \parindent = para indent)
     {\bfseries}% Thm head font
     {}%        Punctuation after thm head
     {1.5ex}%     Space after thm head: " " = normal interword space;
            %   \newline = linebreak
     {\llap{\thmnumber{#2}\hskip2mm}% Thm head spec (empty means `normal')
      \thmname{#1}\thmnote{\bfseries{} #3}}

\newcounter{thm}[section]
\renewcommand{\thethm}{\thesection.\arabic{thm}}

\theoremstyle{mio}
\newtheorem{theorem}[thm]{Theorem}
\newtheorem{corollary}[thm]{Corollary}
\newtheorem{proposition}[thm]{Proposition}
\newtheorem{lemma}[thm]{Lemma}
\newtheorem{fact}[thm]{Fact}
\newtheorem{definition}[thm]{Definition}
\newtheorem{assumption}[thm]{Assumption}
\newtheorem{void_thm}[thm]{}
\theoremstyle{liscio}
\newtheorem{void_def}[thm]{}
\newtheorem{remark}[thm]{Remark}
\newtheorem{notation}[thm]{Notation}
\newtheorem{note}[thm]{Note}
\newtheorem{exercise}[thm]{Exercise}
\newtheorem{example}[thm]{Example}
\setlength{\partopsep}{0mm}
\setlength{\topsep}{0mm}

\def\QED{\noindent\nolinebreak[4]\hspace{\stretch{1}}\rlap{\ \ $\Box$}\medskip}
\renewenvironment{proof}[1][Proof]%
{\begin{trivlist}\item[\hskip\labelsep {\bf #1}]}
{\QED\end{trivlist}}

\newenvironment{void}[1][]%
{\begin{trivlist}\item[\hskip\labelsep {\bf #1}]}
{\QED\end{trivlist}}

\pagestyle{plain}

\definecolor{violet}{RGB}{115, 0, 205}
\definecolor{brown}{RGB}{150, 50, 10}
\definecolor{green}{RGB}{5,110, 35}
\definecolor{emphcolor}{rgb}{.90,.98,.90}

\def\bl{\color{black}}
%\def\bl{\color{brown}}
\def\mr{\color{brown}}
\def\gr{\color{green}}
\def\vl{\color{violet}}

\def\mrA{{\mr\Aa}}
\def\mrB{{\mr\B}}
\def\mrC{{\mr\C}}
\def\mrD{{\mr\D}}
\def\mrE{{\mr\Ee}}
\def\mrG{{\mr\G}}
\def\mrU{{\mr\U}}
\def\mrV{{\mr\V}}
\def\mrS{{\mr\S}}
\def\mrP{{\mr\P}}
\def\mrW{{\mr\W}}
\def\grB{{\gr\B}}
\def\grC{{\gr\C}}
\def\grD{{\gr\D}}
\def\grV{{\gr\V}}

\renewcommand*{\emph}[1]{%
   \kern-0.2ex 
   \smash{\tikz[baseline]
   \node[ rectangle, fill=emphcolor, rounded corners, 
          inner xsep=.3ex, inner ysep=.2ex, anchor=base,
          minimum height = 3ex
         ]{#1};
   }
   \kern-1.2ex 
}

\begin{document}
\raggedbottom
\begin{center}
   {\huge\bfseries Scretch paper\\[3ex] \normalfont\normalsize 
   A.A. \& V.V.\vskip-1ex 
   Università di Torino\vskip-1ex \monthname\ \the\year}
\end{center}

\def\medrel#1{\parbox[t]{6ex}{$\displaystyle\hfil #1$}}

\bigskip\hfil
\parbox{0.9\textwidth}{
   \textbf{Abstract} \ 
   Poche idee, ben confuse.
}
%%%%%%%%%%%%%%%%%%%%%%%%%%%%%%%%%%%%%%%%%%%%%%%%%%%%%%%%%%%%%%%%%%%%%%%%%%%%
%%%%%%%%%%%%%%%%%%%%%%%%%%%%%%%%%%%%%%%%%%%%%%%%%%%%%%%%%%%%%%%%%%%%%%%%%%%%
%%%%%%%%%%%%%%%%%%%%%%%%%%%%%%%%%%%%%%%%%%%%%%%%%%%%%%%%%%%%%%%%%%%%%%%%%%%%
%%%%%%%%%%%%%%%%%%%%%%%%%%%%%%%%%%%%%%%%%%%%%%%%%%%%%%%%%%%%%%%%%%%%%%%%%%%%
%%%%%%%%%%%%%%%%%%%%%%%%%%%%%%%%%%%%%%%%%%%%%%%%%%%%%%%%%%%%%%%%%%%%%%%%%%%%
%%%%%%%%%%%%%%%%%%%%%%%%%%%%%%%%%%%%%%%%%%%%%%%%%%%%%%%%%%%%%%%%%%%%%%%%%%%%
%%%%%%%%%%%%%%%%%%%%%%%%%%%%%%%%%%%%%%%%%%%%%%%%%%%%%%%%%%%%%%%%%%%%%%%%%%%%
%%%%%%%%%%%%%%%%%%%%%%%%%%%%%%%%%%%%%%%%%%%%%%%%%%%%%%%%%%%%%%%%%%%%%%%%%%%%
\section{Introduction}\label{intro}


%%%%%%%%%%%%%%%%%%%%%%%%%%%%%%%%%%%%%%%%%%%%%%%%%%%%%%%%%%%%%%%%%%%%%%%%%%%%
%%%%%%%%%%%%%%%%%%%%%%%%%%%%%%%%%%%%%%%%%%%%%%%%%%%%%%%%%%%%%%%%%%%%%%%%%%%%
%%%%%%%%%%%%%%%%%%%%%%%%%%%%%%%%%%%%%%%%%%%%%%%%%%%%%%%%%%%%%%%%%%%%%%%%%%%%
%%%%%%%%%%%%%%%%%%%%%%%%%%%%%%%%%%%%%%%%%%%%%%%%%%%%%%%%%%%%%%%%%%%%%%%%%%%%
%%%%%%%%%%%%%%%%%%%%%%%%%%%%%%%%%%%%%%%%%%%%%%%%%%%%%%%%%%%%%%%%%%%%%%%%%%%%
%%%%%%%%%%%%%%%%%%%%%%%%%%%%%%%%%%%%%%%%%%%%%%%%%%%%%%%%%%%%%%%%%%%%%%%%%%%%
%%%%%%%%%%%%%%%%%%%%%%%%%%%%%%%%%%%%%%%%%%%%%%%%%%%%%%%%%%%%%%%%%%%%%%%%%%%%
%%%%%%%%%%%%%%%%%%%%%%%%%%%%%%%%%%%%%%%%%%%%%%%%%%%%%%%%%%%%%%%%%%%%%%%%%%%%
\section{Preliminaries}\label{pre}

%%%%%%%%%%%%%%%%%%%%%%%%%%%%%%%%%%%%%%%%%%%%%%%%%%%%%%%%%%%%%%%%%%%%%%%%%%%%
%%%%%%%%%%%%%%%%%%%%%%%%%%%%%%%%%%%%%%%%%%%%%%%%%%%%%%%%%%%%%%%%%%%%%%%%%%%%
%%%%%%%%%%%%%%%%%%%%%%%%%%%%%%%%%%%%%%%%%%%%%%%%%%%%%%%%%%%%%%%%%%%%%%%%%%%%
%%%%%%%%%%%%%%%%%%%%%%%%%%%%%%%%%%%%%%%%%%%%%%%%%%%%%%%%%%%%%%%%%%%%%%%%%%%%
%%%%%%%%%%%%%%%%%%%%%%%%%%%%%%%%%%%%%%%%%%%%%%%%%%%%%%%%%%%%%%%%%%%%%%%%%%%%
%%%%%%%%%%%%%%%%%%%%%%%%%%%%%%%%%%%%%%%%%%%%%%%%%%%%%%%%%%%%%%%%%%%%%%%%%%%%
%%%%%%%%%%%%%%%%%%%%%%%%%%%%%%%%%%%%%%%%%%%%%%%%%%%%%%%%%%%%%%%%%%%%%%%%%%%%
%%%%%%%%%%%%%%%%%%%%%%%%%%%%%%%%%%%%%%%%%%%%%%%%%%%%%%%%%%%%%%%%%%%%%%%%%%%%
\section{Uniform samples}\label{samples}
\def\medrel#1{\parbox[t]{6ex}{\hfil$#1$}}
\def\ceq#1#2#3{\parbox[t]{25ex}{$\displaystyle #1$}\medrel{#2}$\displaystyle  #3$}
\def\uuu{{\mathds 1}}
\def\gruuu{{\mathds 1}}
\def\Av{{\rm Av}}

Fix a tuple of variables $x$ and let $\bar x=\<x_i:i<\omega\>$, where $|x_i|=|x|$.
%
For $p(\bar x)\in S(\U)$ and $\psi(x)\in L(\U)$ we define

\ceq{\hfill\emph{$\Av_{p\restriction n}\,\psi(x)$}}{=}{\frac1n\Big|\big\{i<n\ :\ p(\bar x)\proves\psi(x_i)\big\}\Big|}

If $p(\bar x)$ is the type containing $\bar x=\bar a$ for some $\bar a\in \U^{|x|\cdot\omega}$, we write \emph{$\Av_{\bar a{\restriction n}}\,\psi(x)$.} 

\begin{definition}
We say that $p(\bar x)\in S(\U)$ is a \emph{sample\/} if the limit below exists for every formula $\psi(x)\in L(\U)$.

\ceq{\hfill\emph{$\Av_p\,\psi(x)$}}{=}{\lim_{n\to\infty}\Av_{p\restriction n}\,\psi(x).}

Let $\phi(x,z)\in L$ be given. We say that $p(\bar x)$ is a \emph{$\phi$-sample\/} if the formula $\psi(x)$ above is restricted to range over those of the form $\phi(x,b)$ for $b\in\U^{|z|}$.

We say that $p(\bar x)$ is a \emph{uniform $\phi$-sample\/} if it is a sample and for every $\epsilon>0$ there is an $k$ such that $\Av_{p\restriction n}\,\phi(x,b)$ is within $\epsilon$ from $\Av_p\,\phi(x,b)$ for all $n>k$ and $b\in\U^{|z|}$.

We say that $p(\bar x)$ is a \emph{uniform sample\/} if it is a  uniform $\phi$-sample for all $\phi(x,z)\in L$.\QED
\end{definition}
Define

\ceq{\hfill\emph{$\Av_{p\restriction n}\,q(x)$}}
{=}
{\inf\Big\{\Av_{p\restriction n}\,\psi(x)\ :\ \psi(x)\in q\Big\}}

Clearly the infimum above is attained.

\begin{proposition}
      If $p(\bar x)$ is a uniform sample and $q(x)\subseteq L(\U)$, then the limit below exists and
      
      \ceq{\hfill\lim_{n\to\infty}\Av_{p\restriction n}\,q(x)}
      {=}
      {\inf\Big\{\Av_p\,\phi(x)\ :\ \phi(x)\in q\Big\}}
      
\end{proposition}

\begin{proof}
      
      

      Fix $\epsilon>0$.
      %
      Let $\phi(x)\in q$ be such that $\Av_p\,\phi(x)$ is within $\epsilon$ from the infimum above.
      %
      Let $k$ be such that $\Av_{p\restriction n}\,\phi(x)$ is within $\epsilon$ from  $\Av_p\,\phi(x)$ for every $n>k$.
      %
      Then, for every $n>k$,

      \ceq{\hfill\Av_{p\restriction n}\,q(x)}
      {\le}
      {\Av_{p\restriction n}\,\phi(x)}

      \ceq{}
      {\le}
      {\Av_p\,\phi(x) + \epsilon}

      \ceq{}
      {\le}
      {\inf\Big\{\Av_p\,\phi(x)\ :\ \phi(x)\in q\Big\} + 2\epsilon}

      For the converse inequality, let $\psi_n(x)$ be such that $\Av_p\,\psi_n(x)=\Av_{p\restriction n}\,q(x)$. Then for all $n$

      \ceq{\hfill\inf\Big\{\Av_p\,\phi(x)\ :\ \phi(x)\in q\Big\}}
      {\le}
      {\Av_{p\restriction n}\,\psi(x)}


\end{proof}

\begin{proposition}
      Let $\bar x=\<x_i:i<\omega\>$ and $|x_i|=|x|$ and let $p(\bar x)\in S(\U)$.
      %  
      Then the following are equivalent 
      \begin{itemize}
            \item[1.] $p(\bar x)$ is a uniform global sample;
            \item[2.] for every $\epsilon>0$, there is an $n$ and a formula $\theta(\bar x)\in p$ such that $\Av_n\big(\bar a;\phi(x,b)\big)$ is within $\epsilon$ from $\Av_p\phi(x,b)$ for all $b\in\U^{|z|}$ and all $\bar a\models\theta(\bar x)$.
      \end{itemize}
\end{proposition}

\begin{proof}
???
\end{proof}

Vale anche/solo/nemmeno la versione non uniforme della proposizione?

\begin{corollary}
      Let $p(\bar x)$ bea global sample finitely satisfied in $M$.
      %
      Let $q'(x)=q(x)\cup\{\phi(x)\}$ for $q(x)\in S(M)$ and $\phi(x)\in L(\U)$.
      %
      Then $\Av_p\,q'(x)$ is either $0$ or $1$.
\end{corollary}
\begin{proof}
???
Under the assumptions of the corollary, for every $\epsilon>0$ there is an $n$ and a tuple $\bar a\in M^{|x|\cdot\omega}$ such that $\Av_{\bar a{\restriction n}}\,q'(x)$ is within $\epsilon$ from $\Av_p\,q'(x)$. As $\Av_{\bar a{\restriction n}}\,q'(x)$
\end{proof}

Note that, if $p(\bar x)$ is definable, say over $M$, then there is a formula $\psi_{m/n}(z)\in L(M)$ such that

\ceq{\hfill\psi_{m/n}(z)}{\IFF}{\Av_n\big(p(\bar x);\phi(x,z)\big)=\frac{m}{n}.}



%%%%%%%%%%%%%%%%%%%%%%%%%%%%%%%%%%%%%%%%%%%%%%%%%%%%%%%%%%%%%%%%%%%%%%%%%%%%
%%%%%%%%%%%%%%%%%%%%%%%%%%%%%%%%%%%%%%%%%%%%%%%%%%%%%%%%%%%%%%%%%%%%%%%%%%%%
%%%%%%%%%%%%%%%%%%%%%%%%%%%%%%%%%%%%%%%%%%%%%%%%%%%%%%%%%%%%%%%%%%%%%%%%%%%%
%%%%%%%%%%%%%%%%%%%%%%%%%%%%%%%%%%%%%%%%%%%%%%%%%%%%%%%%%%%%%%%%%%%%%%%%%%%%
%%%%%%%%%%%%%%%%%%%%%%%%%%%%%%%%%%%%%%%%%%%%%%%%%%%%%%%%%%%%%%%%%%%%%%%%%%%%
%%%%%%%%%%%%%%%%%%%%%%%%%%%%%%%%%%%%%%%%%%%%%%%%%%%%%%%%%%%%%%%%%%%%%%%%%%%%
%%%%%%%%%%%%%%%%%%%%%%%%%%%%%%%%%%%%%%%%%%%%%%%%%%%%%%%%%%%%%%%%%%%%%%%%%%%%
\section{References}
\begin{biblist}[]\normalsize

\bib{DK}{book}{
   author={Dodos, Pandelis},
   author={Kanellopoulos, Vassilis},
   title={\href{http://users.uoa.gr/~pdodos/Publications/RT.pdf}
         {Ramsey theory for product spaces}},
   series={Mathematical Surveys and Monographs},
   volume={212},
   publisher={American Mathematical Society},
%   publisher={American Mathematical Society, Providence, RI},
   date={2016},
}
\bib{DZ}{book}{
   author={Zambella, Domenico},
   title={\href{https://github.com/domenicozambella/creche/raw/master/creche.pdf}
         {A cr\`eche course in model theory}},
   date={2018}, 
   series={AMS Open Math Notes}, 
   note={(The link points to the github version)}
}
\end{biblist}

\end{document}
